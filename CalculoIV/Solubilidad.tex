\documentclass[a4paper,12pt]{article}
\usepackage[utf8]{inputenc}
\usepackage[T1]{fontenc}
\usepackage{amssymb}
\usepackage{amsmath}
\usepackage{mathabx}
\usepackage{graphicx}
\usepackage[all]{xy}
\newtheorem{definicion}{Definición}[subsection]
\newtheorem{teoremas}{Teorema}[subsection]
\newtheorem{ejemplos}{Ejemplo}[subsection]
\newtheorem{observ}{Observación}[subsection]
\newtheorem{prop}{Proposición}[subsection]
\newtheorem{Corolario}{Corolario}[subsection]



\begin{document}
\section{Lenguajes formales}


\begin{definicion}
    Sea  \textbf{A}, cierto conjunto abstracto, un alfabeto. Las sucesiones finitas de los elementos de \textbf{A} se denominan
    \textbf{expresiones} en \textbf{A}, las sucesiones finitas de expresiones se denominan, \textbf{textos}.
\end{definicion}

Nosotros hablaremos de un Lenguaje con el alfabeto $A$, si ciertas expresiones y textos han sido distinguidos
(Como `compuestos correctamente', `sensatos', etc.).

En los lenguajes naturales la totalidad de las expresiones y textos
distinguidos suele tener límites vagos. Cuanto más formalizado está lenguaje, tanto más nítidamente están contorneados
estos límites.
Las reglas  de formación de las expresiones y textos distinguidos constituyen  la \textit{sintaxis} del lenguaje.
Las reglas de su confrontación con la realidad conciernen a la \textit{semántica} del mismo.

La descripción de la sintaxis y semántica se realiza por medio del metalenguaje. En la mayoría de los casos eso es un
fragmento del argot mátematico usual.







\end{document}
